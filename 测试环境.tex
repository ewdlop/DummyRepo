\documentclass[12pt]{article}
\usepackage[cm-default]{fontspec}
\usepackage{xunicode}
\usepackage{xltxtra}
\usepackage{amsmath, amssymb, graphicx, geometry, hyperref, chemfig, circuitikz, musicography}
\usepackage{pdflscape} % For landscape pages
\setmainfont[BoldFont=Adobe Heiti Std]{SimSun} % Set main font

\title{\LaTeX Document}
\author{}
\date{}

\begin{document}

\maketitle

\section*{Introduction to \LaTeX}
\LaTeX{} is a document preparation system for the \TeX{} typesetting program. It offers programmable desktop publishing features and extensive facilities for automating most aspects of typesetting and desktop publishing, including numbering and cross-referencing, tables and figures, page layout, bibliographies, and much more. \LaTeX{} was originally written in 1984 by Leslie Lamport and has become the dominant method for using \TeX; few people write in plain \TeX{} anymore. The current version is \LaTeXe.

这是一个测试。 \\
\textbf{测试环境}:XeTeX TeXLive2008

\section*{Mathematical Equations}
\begin{align}
    E &= mc^2 \\
    m &= \frac{m_0}{\sqrt{1-\frac{v^2}{c^2}}}
\end{align}

For all \( x \in \mathbb{R} \), \( f(x) > 0 \):
\[
\forall x \in \mathbb{R}, \, f(x) > 0
\]

Sum of all elements \( x_i \) from 1 to \( n \):
\[
\sum_{i=1}^n x_i
\]

Product of all elements \( x_i \) from 1 to \( n \):
\[
\prod_{i=1}^n x_i
\]

For all elements \( x \) in the set \( A \), \( x^2 > 0 \):
\[
\forall x \in A, \, x^2 > 0
\]

Using a custom "all" command:
\newcommand{\all}{\forall}
\[
\all x \in \mathbb{R}, \, x^2 \geq 0
\]

\section*{Chessboard Example}
\fenboard{%
r5k1/%
1b1p1ppp/%
p7/%
1p1Q4/%
2p1r3/%
PP4Pq/%
BBP2b1P/%
R4R1K w - - 0 20}
\mbox{}\showboard

\section*{Music Example}
\begin{music}
\generalsignature{-1} % One flat
\startextract % Start a line of music
\Notes \csong{紅}\Dqbu gg\en
\Notes \ibu0f0\qb0f\nbbu0\qb0f\tbu0\qb0d\en
\Notes \csong{顏}\qa{fd}\en
\endextract % End a line of music
\end{music}

\section*{Chemical Diagram}
\chemfig{
 H_3C-[:72]{\color{blue}N}*5(- 
*6(-(={\color{red}O})-
{\color{blue}N}(-CH_3)-
(={\color{red}O})-
{\color{blue}N}(-CH_3)-=)--
{\color{blue}N}=-)}

\section*{Electrical Circuit Diagram}
\begin{circuitikz}
\draw
  (0,0) to[C, l=10<\micro\farad>] (0,2) -- (0,3)
        to[R, l=2.2<\kilo\ohm>] (4,3) -- (4,2)
        to[L, l=12<\milli\henry>, i=$i_1$,v=b] (4,0) -- (0,0)
  (0,2) to[R, l=1<\kilo\ohm>, *-] (2,2) 
        to[cV, i=1,v=$\SI{.3}{\kilo\ohm} i_1$] (4,2)
  (2,0) to[I, i=1<\milli\ampere>, -*] (2,2);
\end{circuitikz}

\section*{Poetry in Landscape Mode}
\begin{landscape}
\[
\text{〔■想〕 〔六シ〕 〔■起〕 〔六ˋ、〕 〔■來〕 〔工ˊ〕 〔■那〕 〔ㄨˋ〕}
\]
\end{landscape}

\end{document}
