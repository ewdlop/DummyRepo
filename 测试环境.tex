\documentclass[12pt]{article}
\usepackage[cm-default]{fontspec}
\usepackage{xunicode}
\usepackage{amsmath}
\title{\LaTeX}
\author{}
\date{}
\begin{document}
\maketitle
  \LaTeX{} is a document preparation system
  for the \TeX{}   typesetting program. It offers
  programmable desktop publishing features and
  extensive facilities for automating most aspects
  of typesetting and desktop publishing, including
  numbering and cross-referencing, tables and figures,
  page layout, bibliographies,   and much more.
  \LaTeX{} was originally written in 1984 by Leslie
  Lamport and has become the dominant method for
  using \TeX; few people write in plain \TeX{} anymore.
  The current version is  \LaTeXe.
\setmainfont[BoldFont=Adobe Heiti Std]{SimSun}

这是一个测试。                             \\
\textbf{测试环境}:XeTeX TeXLive2008      \\
  \begin{align}
    E &= mc^2                              \\
    m &= \frac{m_0}{\sqrt{1-\frac{v^2}{c^2}}}
  \end{align}
\end{document}

For all \( x \in \mathbb{R} \), \( f(x) > 0 \):
\[
\forall x \in \mathbb{R}, \, f(x) > 0
\]

\end{document}

\documentclass{article}
\usepackage{amsmath}

\begin{document}

Sum of all elements \( x_i \) from 1 to \( n \):
\[
\sum_{i=1}^n x_i
\]

Product of all elements \( x_i \) from 1 to \( n \):
\[
\prod_{i=1}^n x_i
\]

\end{document}

\documentclass{article}
\usepackage{amsmath}

\begin{document}

For all elements \( x \) in the set \( A \), \( x^2 > 0 \):
\[
\forall x \in A, \, x^2 > 0
\]

\end{document}

\documentclass{article}

% Define a custom command for universal quantification
\newcommand{\all}{\forall}

\begin{document}

Using the custom "all" command:
\[
\all x \in \mathbb{R}, \, x^2 \geq 0
\]

\end{document}

\documentclass{article}
\usepackage{amsmath, amssymb, graphicx, geometry, hyperref}

\begin{document}

A document that includes all commonly needed features.

\end{document}

\normalboard
\begin{position}
\piece{a}{1}{r}
\piece{i}{1}{r}
\piece{b}{1}{n}
\piece{h}{1}{n}
\piece{c}{1}{b}
\end{position}

\fenboard{%
r5k1/%
1b1p1ppp/%
p7/%
1p1Q4/%
2p1r3/%
PP4Pq/%
BBP2b1P/%
R4R1K w - - 0 20}
\mbox{}\showboard

\begin{landscape}
〔■想 〕
〔六シ 〕
〔■起 〕
〔六ˋ、〕
〔■來 〕
〔工ˊ 〕
〔■那 〕
〔ㄨˋ 〕

\begin{music}
\generalsignature{-1}%一個降記號
\startextract%一行譜之開始
\Notes \csong{紅}\Dqbu gg\en
\Notes \ibu0f0\qb0f\nbbu0\qb0f\tbu0\qb0d\en
\Notes \csong{顏}\qa{fd}\en
\endextract%一行譜之結束
\end{music}

\chemfig{
 H_3C-[:72]{\color{blue}N}*5(- 
*6(-(={\color{red}O})-
{\color{blue}N}(-CH_3)-
(={\color{red}O})-
{\color{blue}N}(-CH_3)-=)--
{\color{blue}N}=-)}

\begin{circuitikz}\draw
  (0,0) to[C, l=10<\micro\farad>] (0,2) -- (0,3)
        to[R, l=2.2<\kilo\ohm>] (4,3) -- (4,2)
        to[L, l=12<\milli\henry>, i=$i_1$,v=b] (4,0) -- (0,0)
  (4,2) { to[D*, *-*, color=red] (2,0) }
  (0,2) to[R, l=1<\kilo\ohm>, *-] (2,2) 
        to[cV, i=1,v=$\SI{.3}{\kilo\ohm} i_1$] (4,2)
  (2,0) to[I, i=1<\milli\ampere>, -*] (2,2); 
\end{circuitikz}
\end{landscape}
